\documentclass{article}
\usepackage[utf8]{inputenc}
\usepackage{amsmath}
\usepackage{graphicx}

\title{Teste de Compilação LaTeX}
\author{API Phoenix}
\date{\today}

\begin{document}

\maketitle

\section{Introdução}
Este é um documento de teste para verificar se a rota de compilação LaTeX está funcionando corretamente.

\section{Matemática}
A famosa fórmula de Euler:
\[
e^{i\pi} + 1 = 0
\]

Equação quadrática:
\[
x = \frac{-b \pm \sqrt{b^2 - 4ac}}{2a}
\]

\section{Listas}
\subsection{Lista Numerada}
\begin{enumerate}
    \item Primeiro item
    \item Segundo item
    \item Terceiro item
\end{enumerate}

\subsection{Lista não Numerada}
\begin{itemize}
    \item Item A
    \item Item B
    \item Item C
\end{itemize}

\section{Tabela}
\begin{tabular}{|c|c|c|}
\hline
\textbf{Nome} & \textbf{Idade} & \textbf{Cidade} \\
\hline
João & 25 & São Paulo \\
Maria & 30 & Rio de Janeiro \\
Carlos & 35 & Belo Horizonte \\
\hline
\end{tabular}

\section{Conclusão}
Se este documento foi compilado com sucesso, significa que a API está funcionando perfeitamente!

\textbf{Parabéns!} A compilação foi um sucesso.

\end{document}
